\documentclass[12pt, a4paper]{memoir}
\usepackage[utf8]{inputenc}
\usepackage[brazilian]{babel}
\usepackage{indentfirst}
\usepackage{url}
\usepackage{hyperref}
\usepackage{listings}
\lstdefinestyle{code} {
  language=C++,
  keepspaces=true,
  frame=lines,
  numbers=left,
  columns=flexible
}

\lstdefinestyle{result} {
  basicstyle=\small\ttfamily,
  columns=flexible,
  breaklines=true
}

\title{Linguagem Ultima
  \\ Analisador Sintático}
\author{BELO, Bruno da Silva \and ROCHA, Wesley Marques}
\date{\today}
\newcommand{\institution}{Universidade Federal de Alagoas
  \\ Instituto de Computação}
\newcommand{\department}{Ciência da Computação}

\renewcommand{\maketitlehooka}{
  \centering
  \institution\\
  \emph{\department}\\[.2cm]
  \par
  \hrulefill
  \vfill}
\renewcommand{\maketitlehookb}{\vfill}

\renewcommand{\thesection}{\arabic{section}}
\renewcommand{\thesubsection}{\thesection.\arabic{subsection}}
\makeatletter
\let\l@subsection\l@section
\let\l@section\l@chapter
\makeatother
\setsecnumdepth{subsection}

\begin{document}

\frontmatter
\begin{titlingpage}
  \maketitle
\end{titlingpage}

\tableofcontents

\mainmatter

\section{Códigos-fontes}
\label{sec:codigo_fonte}

\emph{Só foram adicionados os códigos-fontes criados depois do léxico, os códigos-fontes que não estão aqui sofreram nenhuma ou pouca modificação para poder se adequar}
\lstinputlisting[style=code, label=syntax.cpp, caption=syntax.cpp]{../../src/syntax.cpp}

\lstinputlisting[style=code, label=syntax.h, caption=syntax.h]{../../src/syntax.h}

\lstinputlisting[style=code, label=main.cpp, caption=main.cpp]{../../src/main.cpp}

Este arquivo é a tabela do analisador. Cada linha representa uma derivação da tabela, onde o primeiro número é o não-terminal, o segundo é o terminal e o terceiro é a derivação em si, pórem o número da derivação do arquivo é $-1$ o tabela, por conta que o array começa do 0 e não do 1, assim, por exemplo, a D3 é o 2 no arquivo.
\lstinputlisting[style=code, label=syntax.conf, caption=syntax.conf]{../../src/syntax.conf}

\section{Resultados}
\emph{\$ representa a produção vazia}
\subsection{Olá Mundo}
\label{subsec:hello}
\lstinputlisting[style=code, label=hello.res, caption=Resultado de Hello World]{../../build/hello.res}

\subsection{Fibonacci}
\label{subsec:fibonacci}

\lstinputlisting[style=code, label=fibonacci.res, caption=Resultado de Fibonacci]{../../build/fibonacci.res}

\subsection{Shellsort}
\label{subsec:shellsort}

\lstinputlisting[style=code, label=shellsort.res, caption=Resultado de Shellsort]{../../build/shellsort.res}

\section*{Apêndice}
Pelo fato da gramática está mal formulada, alguns erros foram encontrado nela, como:
\begin{lstlisting}[style=code]
  int foo(vector int a:5) {
    int x = 4;
    if (x < foo2(1)) {
      return 2;
    }

    return 0;
  }
\end{lstlisting}
Dado o exemplo acima, há 3 coisas que na gramática ficou devendo.
\begin{enumerate}
\item Em foo(vector int a:5), a gramática não consegue identificar o ``vector int a:5'', ela está identificando só consegue com ``foo(vector a)''.
\item O retorno dentro do ``if'' não é possível na gramática, pois só se foi considerado retorno de fim de função, logo não há uma produção que suporta retornos no meio da função.
\item Em ``if(x < foo2(1))'', a gramática não suporta chamada de funções dentro de expressões.
\end{enumerate}
\end{document}
