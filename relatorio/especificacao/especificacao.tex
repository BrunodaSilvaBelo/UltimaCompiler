\documentclass[a4paper, 11pt, article]{memoir}
\usepackage[utf8]{inputenc}
\usepackage{hyperref}
\usepackage{indentfirst}
\usepackage{listings}
\lstset{language=C++}
\lstset{keepspaces=true}
\lstset{frame=lines}
\lstset{numbers=left}
\usepackage{caption}
\captionsetup[table]{name=Tabela}
\usepackage[brazilian]{babel}

\title{Ultima}
\author{BELO, Bruno da Silva \and ROCHA, Wesley Marques}
\date{\today}
\newcommand{\institution}{Universidade Federal de Alagoas
  \\ Instituto de Computação}
\newcommand{\department}{Ciência da Computação}

\renewcommand{\maketitlehooka}{
  \centering
  \institution\\
  \emph{\department}\\[.2cm]
  \par
  \hrulefill
  \vfill}
\renewcommand{\maketitlehookb}{\vfill}

\renewcommand{\thesection}{\arabic{section}}
\renewcommand{\thesubsection}{\thesection.\arabic{subsection}}
\makeatletter
\let\l@subsection\l@section
\let\l@section\l@chapter
\makeatother
\setsecnumdepth{subsection}

\begin{document}

\frontmatter
\begin{titlingpage}
  \maketitle
\end{titlingpage}

\tableofcontents

\mainmatter

\section{Introdução}
\label{sec:intro}
Ultima se trata de uma linguagem procedural, estaticamente e fortemente tipada.
O programa normalmente consiste de um entry-point, a função main, onde o
algoritmo será escrito, delimitado por um escopo e cada instrução deverá
terminar com um ponto e vírgula.

Será possível a criação de funções para melhorar a legibilidade e
escritabilidade do código. Estas funções devem ser declaradas e definidas, não
somente declarar, antes do entry-point do programa.

Seja na função main ou uma função qualquer criada, o algoritmo será definido
usando declaração e variável, estruturas condicionais e de controle.

\section{Conjunto de Tipo de Dados}
\label{sec:conjunto}

\subsection{ID}
\label{subsec:id}
Um identificador aceito por Ultima é dado da seguinte forma:
\begin{itemize}
\item Inicia-se, obrigatoriamente, com uma letra, maiúscula e minúscula.
\item Os demais caracteres podem ser letras, números ou underscore.
\item Podem possuir qualquer tamanho, porém todo o nome não pode ter uma quebra
  de linha no meio.
\end{itemize}

Exemplo:
\begin{lstlisting}
  int esta_variavel_pode_ser_declarada; # Certo
  int esta_variavel_nao
  _pode_ser_declarada; # Errado
\end{lstlisting}

\subsection{Comentário}
\label{subsec:comentario}
O caractere para expressar um comentário é o ``\#'', assim quando este caractere
for encontrado, o restante da linha depois dele é ignorada.
Ultima não possui comentário de bloco, somente de linha.

\subsection{Inteiro}
\label{subsec:inteiro}
O tipo inteiro é declarado pela palavra reservada ``int'' seguido do ID da
variável. Por ser uma linguagem fortemente tipada, não há coerção entre tipos.

Ela expressa um número inteiro e seus literais são declarado por qualquer
número. As operações para este tipo estão em \ref{subsec:aritmeticos}.

Exemplo:
\begin{lstlisting}
  int var;
  var = 1;
\end{lstlisting}

\subsection{Ponto flutuante}
\label{subsec:float}
O tipo que representa ponto flutuante é declarado utilizando a palavra reservada
``float'' seguido do ID da variável.

Um literal do tipo ponto flutuante é dado por qualquer número separado por
somente um ``.'', assim antes do ``.'' indica o parte inteira e depois dele é a
parte decimal. As operações para este tipo estão em \ref{subsec:aritmeticos}.

Exemplo:
\begin{lstlisting}
  float var;
  var = 1.2;
\end{lstlisting}

\subsection{Caracteres e cadeias de caracteres}
\label{subsec:string}
Ultima tratará cadeias de caracteres usando apenas o tipo ``string'', do mesmo
modo será tratado caracteres isolados. As operações para este tipo estão em
\ref{subsec:concatenacao}.

Exemplo:
\begin{lstlisting}
  string str;
  str = ``Hello World!'';

  string character;
  character = ``A'';
\end{lstlisting}

Dentro de literais de cadeias de caracteres (Delimitados por aspas duplas),
palavras reservadas podem ser usados. Quebra de linha dentro de um literal de
cadeias de caracteres é proibido.

Exemplo:
\begin{lstlisting}
  string str;
  str = ``Esta string nao
  e valida!'';
\end{lstlisting}

\subsection{Booleano}
\label{subsec:booleano}
O tipo booleano é declarado usando a palavra reservada ``bool'' seguido do ID da
variável, os únicos dois possíveis valores para a variável são ``true''
e ``false''. As operações para este tipo estão em \ref{subsec:relacionais} e
\ref{subsec:logicos}.

Exemplo:
\begin{lstlisting}
  bool var;
  var = true;
\end{lstlisting}

\subsection{Arranjos unidimensionais}
\label{subsec:array}
Ultima tratará arranjos unidimensionais como listas com acesso aleatório. Para
serem declaradas, usasse a palavra reservada ``vector'' seguida do tipo que será
armazenado na lista, seu ID e um ``:'' separando o tamanho da lista utilizando
um literal ou variável do tipo inteiro.

Exemplo:
\begin{lstlisting}
  vector int list:3;

  setValueInt(list, index, value);
  getValueInt(list, index);
  removeInt(list, index);
  addInt(list, value); # adiciona no final da lista.
\end{lstlisting}

Há quatro operações nativas para lista. Estas são getValue, setValue, remove e
add cujo final varia a depender do tipo da lista, Int para inteiro, String
para conjunto de caractere, Float para ponto flutuante e Bool para booleano. Com
elas será possível fazer as manipulações das listas.

\section{Conjuntos de Operadores}
\label{sec:operadores}

\subsection{Atribuição}
\label{subsec:atribuicao}
A atribuição é feita pelo operador ``='', onde do lado esquerdo é o endereço na
memória que representa o ID da variável e do lado direito é o valor a ser
guardado por este endereço de memória. Os dois lados devem possuir o mesmo tipo,
pois a linguagem não permite coerção.
Exemplo:
\begin{lstlisting}
  variable = 10;
\end{lstlisting}
Porém não é possível declarar ou atribuir um conjunto de variáveis.

Exemplo:
\begin{lstlisting}
  int a = 1, b = 2, c = 3; # Errado
  int a = 1;
  int b = 2;
  int c = 3; # Certo
\end{lstlisting}

\subsection{Aritméticos}
\label{subsec:aritmeticos}
\begin{itemize}
\item ``+'': Operador binário que retorna a soma dos dois operandos.
\item ``-'': Operador binário que retorna a diferença dos dois operandos.
\item ``*'': Operador binário que retorna a multiplicação dos dois operandos.
\item ``/'': Operador binário que retorna a divisão dos dois operandos.
\item ``\%": Operador binário que retorna o resto da divisão dos dois operandos.
\end{itemize}

O operador unário negativo é ``$\sim$''. Este irá negar valores aritméticos dos
números, tanto inteiro, quanto ponto flutuante.

Exemplo:
\begin{lstlisting}
  int a;
  int b;
  a = a + b;
  b = b - a;
  a = (a * b) / 2;
\end{lstlisting}

\subsection{Relacionais}
\label{subsec:relacionais}
\begin{itemize}
\item ``=='': Operador binário que verifica a igualdade entre dois operandos.
\item ``!='': Operador binário que verifica a desigualdade entre dois operandos.
\item ``\textless'': Operador binário que verifica se o operando da esquerda é
  menor que o operando da direita.
\item ``\textgreater'': Operador binário que verifica se o operando da esquerda
  é maior que o operando da direita.
\item ``\textgreater='': Operador binário que verifica se o operando da esquerda
  é menor ou igual ao operando da direita.
\item ``\textgreater='': Operador binário que verifica se o operando da esquerda
  é maior ou igual ao operando da direita.
\end{itemize}

Exemplo:
\begin{lstlisting}
  if (a > b) {
    # do something
  }
\end{lstlisting}

\subsection{Lógicos}
\label{subsec:logicos}
\begin{itemize}
\item $\lnot$: Operador unário que nega uma expressão lógica.
\item ``\&'': Operador binário que executa um ``and'' lógico.
\item ``$|$'': Operador binário que executa um ``or'' lógico.
\end{itemize}

Exemplo:
\begin{lstlisting}
  if (a > b | c == d) {
    # do something
  }
\end{lstlisting}

\subsection{Concatenação de cadeias de caracteres}
\label{subsec:concatenacao}
A concatenação de cadeias de caracteres será dada pelo operador binário ``+''.
Este por sua vez, criará uma nova string e atribuirá a ela uma união das duas
outros strings.

Exemplo:
\begin{lstlisting}
  string str = ``This str'' + ``More str'';
\end{lstlisting}

\section{Precedência e Associatividade}
\label{sec:precedencia}
Na Tabela \ref{tab:precedencia} mostrará a precedência e associatividade dos
operadores, onde a precedência é maior para os que se encontram mais acima.
\begin{table}[h]
  \centering
  \begin{tabular}{r|r}

    Operadores & Associatividade à \\
    \hline
    () & Não associativo \\
    $\lnot$ & Direita \\
    $\sim$ & Não associativo \\
    $*$ / \% & Esquerda \\
    + - & Esquerda \\
    \textless \textless= \textgreater \textgreater= & Não associativo \\
    == != & Não associativo \\
    \& $|$ & Esquerda \\
    = & Direita

  \end{tabular}
  \caption{Tabela de precedência e associatividade}
  \label{tab:precedencia}
\end{table}

\section{Instruções}
\label{sec:instrucoes}
Os conjuntos dos statements é similar ao C e cada statement é terminado por um
``;'', exceto por if, else, while e for.

\subsection{Estrutura condicional de uma e duas vias}
\label{sub:estruturacondicional}

\subsubsection{if, if-else}
\label{subsubsec:if}
A estrutura condicional ``if'' virá seguida de uma condição lógica dentre de
parenteses e seu escopo é definido por chaves.
O algoritmo irá executar o código contido no ``if'', se e somente se, a sua
condição lógica for verdadeira, caso essa condição não seja satisfeita, é
possível criar, opcionalmente, uma clausula ``else'', que será executada, se e
somente se, a condição do ``if'' em conjunto for falsa.
Exemplo:
\begin{lstlisting}
  int a = 10;
  int b = a;
  if (a == b) {
    # do something
  } else {
    # do anotherthing
  }
\end{lstlisting}

\subsection{Estrutura iterativa com controle lógico}
\label{subsec:iterativalogico}

\subsubsection{while}
\label{subsubsec:while}
``while'' é usado como uma repetição condicional, isto é, a repetição só irá
parar se a sua condição for falsa.

Exemplo:
\begin{lstlisting}
  int a = 0;
  while (a < 10) {
    # do something
    a = a + 1;
  }
\end{lstlisting}

\subsection{Estrutura iterativa controlada por contador}
\label{subsec:iterativacontador}

\subsubsection{for}
\label{subsubsec:for}
O ``for'' será uma estrutura de repetição que receberá três parâmetros: índice,
limite e passo. Ele irá repetir o bloco de código desejado no intervalo
[\emph{índice}, \emph{limite}) variando em \emph{passo} até todo intervalo
ser passado.

Exemplo:
\begin{lstlisting}
  for (int i = 0; 10; 1) { # i ira variar de 0 a 9 ao passo de 1
    # do something
  }
\end{lstlisting}

\subsection{Entrada e saída}
\label{subsec:io}
Ultima terá seis funções nativas responsáveis por entrada e saída, essas seriam:

Para entrada:
\begin{itemize}
\item inputString(string str)
\item inputInt(int i)
\item inputFloat(float f)
\end{itemize}

Para saída:
\begin{itemize}
\item outputString(string str)
\item outputInt(int i)
\item outputFloat(float f)
\end{itemize}

\subsection{Funções}
\label{subsec:function}
Ultima não terá suporte para sobrecarga de funções. Para declarar uma função
define-se o seu tipo, nome da função e parâmetros com seus tipos dentro de
parênteses.
Para chamá-la, basta colocar o nome da função e passar seus parâmetros e o
retorno é feito pela palavra reservada ``return''.

Exemplo:
\begin{lstlisting}
  int sumInt(int x, int y) {
    return x + y;
  }

  float sumFloat(float x, float y) {
    return x + y;
  }

  void sort(vector int list, int size) {
    # do shell sort
  }

  vector int apply_exp(vector int list, int size) {
    # do median
  }

  int a;
  float b;

  a = sumInt(5, 2);
  b = sumFloat(2.4, ~6.3);
\end{lstlisting}

Para Arranjos unidimensionais, a passagem de parâmetros é feita por referência,
os outros tipos são por valor.

Para funções há um tipo especial chamado ``void'', em que significa que tal
função retornará nenhum tipo. Não é possível uma variável ser do tipo ``void''.

Já para conjunto unidimensionais, é necessário colocar a palavra ``vector'' para
indicar que uma lista será passada como parâmetro ou retornada.
\section{Exemplos de Códigos}
\label{sec:code}

\subsection{Olá mundo}
\label{subsec:helloworld}
\begin{lstlisting}
  int main() {
    outputString(``Hello World!'');

    return 0;
  }
\end{lstlisting}

\subsection{Fibonacci}
\label{subsec:fibonacci}
\begin{lstlisting}
  int fibonacci(int n) {
    int f1 = 0;
    int f2 = 1;
    int fi = 0;

    if (n < 0) {
      return 0;
    }

    outputInt(0);
    outputString(``, ``);
    outputInt(1);

    if (n == 0 | n == 1) {
      return 1;
    }

    while(fi < n) {
      fi = f1 + f2;
      f1 = f2;
      f2 = fi;
      outputString(``, ``);
      outputInt(fi);
    }

    return fi;
  }

  int main() {
    int n;
    inputInt(n);

    int fib = fibonacci(n);
    # do something with fib
    return 0;
  }
\end{lstlisting}

\subsection{Shell sort}
\label{subsec:shelsort}
\begin{lstlisting}
  void shellsort(vector int vet, int size) {
    int value;
    int gap = 1;
    while(gap < size) {
      gap = 3 * gap + 1;
    }

    while(gap > 1) {
      gap = gap / 3;
      for (int i = gap; i < size; i = i + 1) {
        value = getValue(vet, i);
        int j = i - gap;

        while(j >= 0 & value < getValue(vet, j)) {
          setValue(vet, j + gap, getValue(vet, j));
          j = j - gap;
        }

        setValue(vet, j + gap, value);
      }
    }
  }

  int main() {
    int size;
    inputInt(size);

    int vet:size;
    for (int i = 0; size; 1) {
      int x;
      inputInt(x);
      add(vet, x);
    }

    shellsort(vet, size);

    return 0;
  }
\end{lstlisting}
\end{document}
