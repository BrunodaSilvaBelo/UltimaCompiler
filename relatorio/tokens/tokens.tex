\documentclass[12pt, a4paper]{IEEEtran}
\usepackage[utf8]{inputenc}
\usepackage[brazilian]{babel}
\usepackage{indentfirst}

\author{BELO, Bruno da Silva \and ROCHA, Wesley Marques}
\title{Linguagem Ultima
  \\ Especificação dos tokens}

\begin{document}
\maketitle

O compilador será feito na linguagem C++.
\begin{itemize}
\item \textbf{Caracteres}
  \begin{itemize}
  \item   \emph{letter} = "A" $|$ "B" $|$ "C" $|$ "D" $|$ "E" $|$ "F" $|$ "G"
    $|$ "H" $|$ "I" $|$ "J" $|$ "K" $|$ "L" $|$ "M" $|$ "N"
    $|$ "O" $|$ "P" $|$ "Q" $|$ "R" $|$ "S" $|$ "T" $|$ "U"
    $|$ "V" $|$ "W" $|$ "X" $|$ "Y" $|$ "Z" $|$ "a" $|$ "b"
    $|$ "c" $|$ "d" $|$ "e" $|$ "f" $|$ "g" $|$ "h" $|$ "i"
    $|$ "j" $|$ "k" $|$ "l" $|$ "m" $|$ "n" $|$ "o" $|$ "p"
    $|$ "q" $|$ "r" $|$ "s" $|$ "t" $|$ "u" $|$ "v" $|$ "w"
    $|$ "x" $|$ "y" $|$ "z";

  \item \emph{digit} = "0" $|$ "1" $|$ "2" $|$ "3" $|$ "4" $|$ "5" $|$ "6"
    $|$ "7" $|$ "8" $|$ "9";

  \item \emph{alpha} = \emph{letter} $|$ \emph{digit};
  \item \emph{all\_characters} = ? Todos caracteres visíveis ?;
  \end{itemize}
\end{itemize}

\begin{itemize}
\item \textbf{Operadores Aritméticos}
  \begin{itemize}
  \item \emph{add\_o} = ``+'' $|$ ``-'';
  \item \emph{mult\_o} = ``*'' $|$ ``/'' $|$ ``\%";
  \item \emph{inv\_o} = ``$\sim$'';
  \end{itemize}
\end{itemize}

\begin{itemize}
\item \textbf{Operadores de Atribuição}
  \begin{itemize}
  \item \emph{atr\_o} = ``='';
  \end{itemize}
\end{itemize}

\begin{itemize}
\item \textbf{Operadores Relacionais}
  \begin{itemize}
  \item \emph{r\_o} = ``\textless'' $|$ ``\textless='' $|$ ``\textgreater''
    $|$ ``\textgreater='';
  \item \emph{re\_o} = ``=='' $|$ ``!='';
  \end{itemize}
\end{itemize}

\begin{itemize}
\item \textbf{Operadores Lógicos}
  \begin{itemize}
  \item \emph{or\_o} = ``$|$'';
  \item \emph{and\_o} = ``\&'';
  \item \emph{neg\_o} = ``$\lnot$'';
  \end{itemize}
\end{itemize}

\begin{itemize}
\item \textbf{Delimitadores de Escopo e Símbolos}
  \begin{itemize}
  \item \emph{open\_paren} = ``(``;
  \item \emph{close\_paren} = ``)'';
  \item \emph{open\_brace} = ``\{``;
  \item \emph{close\_brace} = ``\}'';
  \item \emph{semicolon} = ``;'';
  \item \emph{colon} = ``:'';
  \item \emph{quotation} = `` " ``;
  \item \emph{comma} = ``,'';
  \end{itemize}
\end{itemize}

\begin{itemize}
\item \textbf{Comentário}
  \begin{itemize}
  \item \emph{commentary} = ``\#'';
  \end{itemize}
\end{itemize}

\begin{itemize}
\item \textbf{Retorno de Função}
  \begin{itemize}
  \item \emph{return} = ``return'';
  \end{itemize}
\end{itemize}
\begin{itemize}
\item \textbf{Tipos e seus Literais}
  \begin{itemize}
  \item \textsf{ID}
    \begin{itemize}
    \item \emph{id\_t} = \emph{letter} \{``\_'' $|$ \emph{alpha}\};
    \end{itemize}
  \end{itemize}
  \begin{itemize}
  \item \textsf{Inteiro}
    \begin{itemize}
    \item \emph{int\_t} = ``int'';
    \item \emph{int\_l} = \emph{digit} \{\emph{digit}\};
    \end{itemize}
  \end{itemize}
  \begin{itemize}
  \item \textsf{Float}
    \begin{itemize}
    \item \emph{float\_t} = ``float'';
    \item \emph{float\_l} = \emph{digit} \{\emph{digit}\} ``.'' \emph{digit}
      \{\emph{digit}\};
    \end{itemize}
  \end{itemize}
  \begin{itemize}
  \item \textsf{String}
    \begin{itemize}
    \item \emph{string\_t} = ``string'';
    \item \emph{string\_l} = \emph{quotation} {\emph{all\_characters}}
      \emph{quotation};
    \end{itemize}
  \end{itemize}

  \begin{itemize}
  \item \textsf{Booleano}
    \begin{itemize}
    \item \emph{bool\_t} = ``bool'';
    \item \emph{bool\_l} = ``true'' $|$ ``false'';
    \end{itemize}
  \end{itemize}

  \begin{itemize}
  \item \textsf{Lista}
    \begin{itemize}
    \item \emph{vector\_t} = ``vector'';
    \end{itemize}
  \end{itemize}
\end{itemize}

\begin{itemize}
\item \textbf{Comandos}
  \begin{itemize}
  \item \textsf{Iteração por controle lógico}
    \begin{itemize}
    \item \emph{while\_c} = ``while'';
    \end{itemize}
  \end{itemize}

  \begin{itemize}
  \item \textsf{Iteração por contador}
    \begin{itemize}
    \item \emph{for\_c} = ``for'';
    \end{itemize}
  \end{itemize}

  \begin{itemize}
  \item \textsf{Condicionais de uma ou duas vias}
    \begin{itemize}
    \item \emph{if\_c} = ``if'';
    \item \emph{else\_c} = ``else'';
    \end{itemize}
  \end{itemize}
\end{itemize}
\end{document}
